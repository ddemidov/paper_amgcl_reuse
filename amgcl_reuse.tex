\documentclass[
11pt,%
tightenlines,%
twoside,%
onecolumn,%
nofloats,%
nobibnotes,%
nofootinbib,%
superscriptaddress,%
noshowpacs,%
centertags]%
{revtex4}

\usepackage{ljm}
\usepackage{graphicx}

\begin{document}

\title{Efficient Solution of Non-Steady State Problems with Partial Reuse of
AMG Hierarchy}
\author{\firstname{D.E.}~\surname{Demidov}}
\affiliation{
    Kazan Branch of Joint Supercomputer Center, Scientific Research Institute of System Analysis,
    the Russian Academy of Sciences; 2/31, Lobachevskii str., Kazan 420111 Russia}

\begin{abstract}
Abstract
\end{abstract}

\subclass{35-04, 65-04, 65Y05, 65Y10, 65Y15, 97N80}
\keywords{AMG, partial reuse, non-steady state.}

\maketitle

\section{Introduction}

Most of the numerical simulation problems today involve solution of large
sparse linear systems obtained from discretization of partial differential
equations on either structured or unstructured meshes. The combination of a
Krylov subspace method with algebraic multigrid (AMG) as a preconditioner is
considered to be one the most effective choices for solution of such
systems~\cite{brandt1985algebraic,ruge1987algebraic,Trottenberg2001}. One
disadvantage of the AMG preconditioner is the high cost of its setup. Depending
on the convergence rate of the iterative solution, it may take more than 50\%
of the total compute time. The cost is unavoidable when a steady state problem
is being solved, but it could be amortized for non-steady state problems either
by reusing the complete preconditioner on new time steps, or by partial updates
to the preconditioner using the new system matrix.

Reusing the full preconditioner may work when the system matrix changes slowly
over the time. In this case there is a strong chance that a preconditioner
constructed for a specific time step will act as a reasonably good
preconditioner for a couple of subsequent time steps. However, the deteriorated
quality of the preconditioner may result in a convergence rate that is bad
enough to neglect any time savings obtained by reusing the preconditioner. It
was shown in~\cite{Demidov2012} that this strategy may only be beneficial when
the solution phase is comparable to or is cheaper than the setup phase (for
example, when the solution phase is accelerated by using a GPU).

This paper considers the effects of partial reuse of AMG strategy for solution
of non-steady state problems on the example of the open-source AMGCL
library~\cite{Demidov2019, Demidov2020}.

\section{Partial reuse of AMG hierarchy}

\section{Numerical experiments}

\section{Conclusion}

\bibliographystyle{plain}
\bibliography{ref}	

\end{document}

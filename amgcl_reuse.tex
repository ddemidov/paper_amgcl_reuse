\documentclass[
11pt,%
tightenlines,%
twoside,%
onecolumn,%
nofloats,%
nobibnotes,%
nofootinbib,%
superscriptaddress,%
noshowpacs,%
centertags]%
{revtex4}

\usepackage{ljm}
\usepackage{graphicx}
\usepackage{algorithm}
\usepackage{algorithmic}
\usepackage{cleveref}
\usepackage{tabularx}

\newcolumntype{Y}{>{\raggedleft\arraybackslash}X}

\begin{document}

\title{Reusing AMG Hierarchy for Solution of Non-Steady State Problems}
\author{\firstname{D.~E.}~\surname{Demidov}}
\affiliation{
    Kazan Branch of Joint Supercomputer Center, Scientific Research Institute of System Analysis,
    the Russian Academy of Sciences; 2/31, Lobachevskii str., Kazan 420111 Russia}

\begin{abstract}
    The work considers the full reuse and the partial reuse strategies for the
    algebraic multigrid (AMG) setup cost amortization for the case of a
    non-steady state problem solution. It is shown on the model example of a
    dam break fluid dynamics problem that the partial reuse strategy is able to
    consistently reduce the cost of the AMG setup across multiple time steps.
    The strategy may save up to 50\% of the setup time and, depending on the
    relative cost of the setup, up to 20\% of the total compute time.
\end{abstract}

\subclass{35-04, 65-04, 65Y05, 65Y10, 65Y15, 97N80}
\keywords{AMG, partial reuse, non-steady state.}

\maketitle

\section{Introduction}

Most of the numerical simulation problems today involve solution of large
sparse linear systems obtained from discretization of partial differential
equations on either structured or unstructured meshes. The combination of a
Krylov subspace method with algebraic multigrid (AMG) as a preconditioner is
considered to be one the most effective choices for solution of such
systems~\cite{brandt1985algebraic,ruge1987algebraic,Trottenberg2001}. One
disadvantage of the AMG preconditioner is the high cost of its setup. Depending
on the convergence rate of the iterative solution, the setup may take more than
50\% of the total compute time. This cost is unavoidable when a steady state
problem is being solved, but it could be amortized for non-steady state
problems either by reusing the complete preconditioner on new time steps, or by
partial updates to the preconditioner using the new system matrix.

In this paper two possible strategies of amortizing the cost of the AMG setup
are considered. The first one is the \emph{full reuse} strategy, described
in~\cite{Demidov2012}, where the AMG preconditioner is used completely
unaltered for as long as the solution is able to converge in a reasonable
number of iterations.  Reusing the preconditioner this way may work when the
system matrix changes slowly over the time. In this case there is a strong
chance that a preconditioner constructed for a specific time step will act as a
reasonably good preconditioner for a couple of subsequent time steps. However,
the applicability of the preconditioner may deteriorate with time and may
result in a convergence rate that is bad enough to neglect any time savings due
to reusing the preconditioner. It was shown in~\cite{Demidov2012} that this
strategy is mostly beneficial when the solution phase is comparable to or is
cheaper than the setup phase. This may be the case when the solution phase is
accelerated by using a GPU, or when just a couple of iterations are required
for the solution to converge.

The second approach considered here is the \emph{partial reuse} strategy, which
was recently implemented in the opensource AMGCL library~\cite{Demidov2019,
Demidov2020}. Here, the AMG hierarchy is partially updated on each of the time
steps. Namely, the transfer operators (restriction and interpolation) are
reused for a number of time steps, while the system matrices and the smoother
operators on each level of the hierarchy are updated on every step.  It is
shown that the partial reuse strategy is able to consistently reduce both the
setup cost and the overall compute time, while the full reuse strategy may be
either more efficient or counterproductive, depending on the problem that is
being solved.

The rest of the paper is structured as follows. \Cref{sec:amg} provides an
overview of the AMG method. \Cref{sec:strategies} describes the AMG setup cost
amortization strategies considered in the paper. \Cref{sec:experiments}
describes the numerical experiments used to test the efficiency of the
amortization strategies, and presents the results of the numerical experiments.

\section{Algebraic multigrid} \label{sec:amg}

This section describes the basic principles behind the
AMG~\cite{brandt1985algebraic, Stuben1999}.  The method solves a system of
linear algebraic equations \begin{equation} \label{eq:auf} Au = f,
\end{equation} where $A$ is a square matrix. Mutigrid methods are based on
recursive application of a two-grid scheme, which combines \emph{relaxation}
and \emph{coarse grid correction}. Relaxation, or smoothing iteration $S$, is a
simple iterative method, such as a damped Jacobi or a Gauss--Seidel
iteration~\cite{barrett1994templates}. Coarse grid correction solves the
residual equation on a coarser grid, and improves the fine-grid approximation
with the interpolated coarse-grid solution. Transfer between the grids is
described with the \emph{transfer operators} $P$ (\emph{prolongation} or
\emph{interpolation}) and $R$ (\emph{restriction}).

\begin{algorithm}
    \caption{AMG setup}
    \begin{algorithmic}[1] \label{alg:setup}
        \STATE Start with a system matrix $A_1 \leftarrow A$.
        \WHILE{the matrix $A_i$ is too big to be solved directly}
            \STATE Construct the transfer operators $P_i$ and $R_i$.
            \STATE Construct the smoother $S_i$.
            \STATE Construct the coarser system using Galerkin operator: $A_{i+1}
                \leftarrow R_i A_i P_i$.
        \ENDWHILE
        \STATE Construct a direct solver for the coarsest system $A_L$.
    \end{algorithmic}
\end{algorithm}

In geometric multigrid methods the matrices $A_i$ and operators $P_i$ and $R_i$
are usually supplied by the user based on the problem geometry. In algebraic
multigrid methods the grid hierarchy and the transfer operators are in general
constructed automatically, based only on the algebraic properties of the
matrix~$A$. \Cref{alg:setup} describes the \emph{setup} phase of a generic AMG
method. Here, the transfer operators $P_i$ and $R_i$, as well as the smoother
$S_i$ are constructed from the system matrix $A_i$ on each level of the AMG
grid hierarchy (a common choice for the restriction operator $R$ is the
transpose of the prolongation operator $R=P^T$). Note that the next level in
the AMG hierarchy is fully defined by the transfer operators $P_i$ and $R_i$.
The most time consuming steps of the setup are the transfer operators
construction and the evaluation of the Galerkin operator.

\begin{algorithm}
    \caption{AMG V-cycle}
    \begin{algorithmic}[1] \label{alg:vcycle}
        \STATE Start at the finest level with an initial approximation
        $u_1 \leftarrow u^0$.
        \WHILE{not converged}
            \FOR{each level of the hierarchy, finest-to-coarsest}
                \STATE Apply a couple of smoothing iterations to the current
                       solution: $u_i \leftarrow S_i(f_i, u_i)$.
                \STATE Find residual $e_i \leftarrow f_i - A_i u_i$ and
                       restrict it to the RHS on the coarser level:
                       $f_{i+1} \leftarrow R_i e_i$.
            \ENDFOR
            \STATE Solve the coarsest system directly:
                   $u_L \leftarrow A_L^{-1} f_L$.
            \FOR{each level of the hierarchy, coarsest-to-finest}
                \STATE Update the current solution with the interpolated
                       solution from the coarser level:
                       $u_i \leftarrow u_i + P_i u_{i+1}$.
                \STATE Apply a couple of smoothing iterations to the current
                       solution: $u_i \leftarrow S_i(f_i, u_i)$.
            \ENDFOR
        \ENDWHILE
    \end{algorithmic}
\end{algorithm}

After the AMG hierarchy has been constructed, it may be used to solve the
system using a simple V-cycle shown in \Cref{alg:vcycle}. Usually AMG is not
used standalone, but as a preconditioner with an iterative Krylov subspace
method. In this case a single V-cycle is used as a preconditioning step.

\section{AMG reuse strategies} \label{sec:strategies}

Here the AMG setup cost amortization strategies considered in
the paper are described. 

\begin{algorithm}
    \caption{Full rebuild of the AMG hierarchy}
    \begin{algorithmic}[1]
        \FOR{all time steps}
            \IF{AMG hierarchy has to be rebuilt}
                \STATE Rebuild AMG hierarchy using the current system matrix
            \ENDIF
            \STATE Solve the current system using the latest AMG hierarchy
        \ENDFOR
    \end{algorithmic}
\end{algorithm}

\begin{algorithm}
    \caption{Partial rebuild of the AMG hierarchy}
    \begin{algorithmic}[1]
        \FOR{all time steps}
            \IF{AMG hierarchy has to be rebuilt}
                \STATE Rebuild AMG hierarchy using the current system matrix
            \ELSE
                \STATE \emph{Update the latest AMG hierarchy:}
                \FOR{all levels $i$}
                    \STATE $A_{i+1} \leftarrow R_i A_i P_i$
                    \STATE $S_{i} \leftarrow S(A_i)$
                \ENDFOR
                \STATE Construct a direct solver for the coarsest system $A_L$.
            \ENDIF
            \STATE Solve the current system using the latest AMG hierarchy
        \ENDFOR
    \end{algorithmic}
\end{algorithm}

\section{Numerical experiments} \label{sec:experiments}

In order to test the cost amortization strategies described in the above
section, a model two-fluid dam break scenario was modelled using the Kratos
Multi-Phisics framework~\cite{Dadvand2010,Dadvand2013}. The source code for the
example is available in~\cite{dambreak}. In the scenario, a water-filled cuboid
is positioned in one part of the domain. It is released at the start time and
the water spreads across the domain driven by gravity. The water hits an
obstacle and splashes develop. More details on the boundary conditions and
problem settings can be found in~\cite{Larese2008, Coppola2011}. There are
two basic substeps needed to evolve the solution from time step $n$ to time
step $n+1$:
\begin{enumerate}
    \item Find the motion in both phases as the solution of the two-fluid
        Navier-Stokes equations;
    \item Determine the position of the interface by solving a convection
        equation for the level-set function.
\end{enumerate}

The velocity and pressure fields of two incompressible fluids moving in the
domain $\Omega$ can be described by the incompressible
two-fluid Navier-Stokes equations:
\begin{gather}
    \rho \left[ \frac{\partial \mathbf u}{\partial t} + (\mathbf u \cdot \nabla)
    \mathbf u \right] - \nabla \cdot \left[ 2 \mu \mathbf \varepsilon(\mathbf u) \right]
    + \nabla p = \mathbf f, \\
    \nabla \cdot \mathbf u = 0,
\end{gather}
where $\rho$ is the density, $\mathbf u$ is the velocity field, $\mu$ is the
dynamic viscosity, $p$ is the pressure, $\mathbf \varepsilon(\cdot)$ is the
symmetric gradient operator, and $\mathbf f$ is the external body force vector,
which includes the gravity and bouyancy forces, if required. The density,
velocity, dynamic viscosity, and pressure are defined as
\begin{equation}
    \mathbf u, p, \rho, \mu = \begin{cases}
        \mathbf u_1, p_1, \rho_1, \mu_1 \quad \mathbf x \in \Omega_1, \\
        \mathbf u_2, p_2, \rho_2, \mu_2 \quad \mathbf x \in \Omega_2,
    \end{cases}
\end{equation}
where $\Omega_1$ and $\Omega_2$ indicate the parts of $\Omega$ occupied by
fluids number 1 and 2 correspondingly.

Regarding the second step, the evolution of the fluid interface is updated
using the so-called level set method, which has been widely used to track free
surfaces in mould filling and other metal forming processes. The basic idea of
the level set method is to define a smooth scalar function $\psi(x, t)$, over
the computational domain $\Omega$ that determines the extent of subdomains
$\Omega_1$ and $\Omega_2$. For instance,
positive values may be assigned to the points belonging to $\Omega_1$, and
negative values~--- to the points belonging to $\Omega_2$. The position of the
fluid front will be defined by the iso-value contour $\psi(x, t) = 0$.
The evolution of the front $\psi = 0$ in any control volume $V_t \subset
\Omega$ that is moving with a divergence-free velocity field $\mathbf u$ leads to
\begin{equation}
    \frac{\partial \psi}{\partial t} + (\mathbf u \cdot \nabla) \psi = 0.
\end{equation}

Both the Navier–Stokes and the level set equations are solved using a finite
element model based on a stabilized finite element method, using linear P1
elements for all the unknowns. The level set equation has 104~401 unknowns and
1~331~279 non-zero elements in the system matrix. In this work, the two sets of the equations are
treated as independent non-steady state problems in order to test the
efficiency of the considered AMG setup cost amortization strategies.

\begin{table}
    \caption{Relative cost of the setup operations from \Cref{alg:setup}}
    \centering
    \begin{tabularx}{0.8\textwidth}{l|YY}
        Step & Level set & Navier-Stokes \\
        \hline
        Transfer operators $P_i$ and $R_i$        & 48\% & 21\% \\
        Galerkin operator $A_{i+1} = R_i A_i P_i$ & 29\% & 24\% \\
        Smoother $S_i$                            &  3\% &  4\% \\
        Direct solver for the coarsest system     &  1\% &  0\% \\
    \end{tabularx}
\end{table}

\begin{table}
    \caption{Cost savings for different reuse strategies}
    \centering
    \begin{tabularx}{\textwidth}{l|YYYYYY}
        Strategy & Setup (s) & Solve (s) & Rebuilds
        & Average iterations & Total \newline speedup (\%)
        & Setup \newline speedup (\%) \\
        \hline
        & \multicolumn{6}{c}{Level set, OpenMP} \\
        No reuse      & 1.145 & 2.780 & 49 & 5.0 &      &         \\
        Full reuse    & 0.026 & 3.186 & 1  & 5.5 & 22\% & 4~303\% \\
        Partial reuse & 0.703 & 2.769 & 5  & 5.0 & 13\% & 63\%    \\
        \hline
        & \multicolumn{6}{c}{Level set, CUDA} \\
        No reuse      & 1.890 & 0.781 & 49 & 5.0 &       &         \\
        Full reuse    & 0.039 & 0.897 & 1  & 5.5 & 185\% & 4~746\% \\
        Partial reuse & 1.284 & 0.781 & 5  & 5.0 & 29\%  & 47\%    \\
        %\hline
        %& \multicolumn{6}{c}{Redistancing, OpenMP} \\
        %No reuse      & 1.246 & 9.508  & 49 & 15.7 &      &       \\
        %Full reuse    & 0.245 & 11.324 & 9  & 18.1 & -7\% & 408\% \\
        %Partial reuse & 0.749 & 9.308  & 5  & 15.6 & 7\%  & 66\%  \\
        %\hline
        %& \multicolumn{6}{c}{Redistancing, CUDA} \\
        %No reuse      & 1.904 & 2.636 & 49 & 15.7 &      &       \\
        %Full reuse    & 0.355 & 3.049 & 9  & 18.0 & 33\% & 436\% \\
        %Partial reuse & 1.340 & 2.613 & 5  & 15.6 & 15\% & 42\%  \\
        \hline
        & \multicolumn{6}{c}{Navier-Stokes, OpenMP} \\
        No reuse      & 7.926 &  71.052 & 49 & 16.7 &       &       \\
        Full reuse    & 3.859 & 239.322 & 25 & 45.6 & -68\% & 105\% \\
        Partial reuse & 6.011 &  71.763 &  5 & 16.9 &   2\% &  32\% \\
        \hline
        & \multicolumn{6}{c}{Navier-Stokes, CUDA} \\
        No reuse      & 13.789 & 21.398 & 49 & 16.7 &       &       \\
        Full reuse    &  6.863 & 59.200 & 25 & 46.1 & -47\% & 101\% \\
        Partial reuse & 11.209 & 21.597 &  5 & 16.9 &   7\% &  23\% \\
    \end{tabularx}
\end{table}

\section{Conclusion} \label{sec:conclusion}

\bibliographystyle{spmpsci}
\bibliography{ref}

\end{document}
